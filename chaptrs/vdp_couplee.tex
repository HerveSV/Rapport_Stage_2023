\chapter{Étude de deux oscillateurs Van der Pol couplés}
%
On étudie le système suivant, composé de deux oscillateurs de Van der Pol identiques couplé par une force de rappel linéaire relativement faible.
%
\begin{equation}
  \left\{\begin{array}{@{}l@{}}
    \ddot{x}_1 + x_1  + \epsilon(x_1^2 - 1)\dot{x}_1 = \epsilon k(x_2 - x_1) \\
    \\
    \ddot{x}_2 + x_2 + \epsilon(x_2^2 - 1)\dot{x}_2 = \epsilon k(x_1 - x_2)
  \end{array}\right.\,
  \label{eq:coupled_vdp}
\end{equation}
%
Encore une fois, on procède par la méthode de moyennement pour obtenir une approximation de l'enveloppe lentement variable :
%
\begin{equation}
    2\dot{z}_1(t) = 
     \epsilon \left[ z_1(t) - |z_1|^2z_1(t) \right]
    - i\epsilon k\left( z_2(t) - z_1(t)\right)
\end{equation}
%
On sépare les parties réelles et imaginaires et on obtient un système d'équation pour $\dot{r}_1$,  $\dot{r}_2$,  $\dot{\phi}_1$ et  $\dot{\phi}_2$. %le système :
%
%
%   \begin{equation}
%       \left\{\begin{array}{@{}l@{}}
%         \dot{r}_1 = \frac{\epsilon}{8} r_1 \left( 4 - r_2^2 \right) + k\frac{r_2}{2}\sin(\phi_2 - \phi_1)\\
%         \\
%         \dot{r}_2 = \frac{\epsilon}{8} r_2 \left( 4 - r_2^2 \right) + k\frac{r_1}{2}\sin(\phi_1 - \phi_2)\\
%         \\
%         \dot{\phi}_1 = \frac{k}{2}\left( 1 - \frac{r_2}{r_2}\cos(\phi_2 - \phi_1)\right) \\
%         \\
%         \dot{\phi}_2 = \frac{k}{2}\left( 1 - \frac{r_1}{r_2}\cos(\phi_1 - \phi_2)\right)
%       \end{array}\right.\,
%    \end{equation}
%
%
Or, l'oscillateur de Van der Pol n'ayant pas de phase de référence (sa phase n'étant pas fixée par une force exterieure), 
%la phase des deux oscillateurs peuvent se synchroniser. En effet, 
l'évolution du système ne dépend que de la différence entre les phases des oscillateurs. 
On peut donc réduire le système à un système de trois équations en introduisant la variable de déphasage $\Theta = \phi_2 - \phi_1$ :
%
\begin{equation}
    \left\{\begin{array}{@{}l@{}}
        \dot{r}_1 =  \frac{\epsilon}{8} r_1 \left( 4 - r_2^2 \right) + \epsilon k\frac{r_2}{2}\sin(\phi_2 - \phi_1)\\
        \\
        \dot{r}_2 =  \frac{\epsilon}{8} r_2 \left( 4 - r_2^2 \right) + \epsilon k\frac{r_1}{2}\sin(\phi_1 - \phi_2)\\
        \\
        \dot{\phi}_1 = \epsilon \frac{k}{2}\left( 1 - \frac{r_2}{r_1}\cos(\phi_2 - \phi_1)\right) \\
        \\
        \dot{\phi}_2 = \epsilon \frac{k}{2}\left( 1 - \frac{r_1}{r_2}\cos(\phi_1 - \phi_2)\right)
    \end{array}\right.\,
    \implies
    \left\{\begin{array}{@{}l@{}}
        \dot{r}_1 = \frac{\epsilon}{8} r_1 \left( 4 - r_2^2 \right) + \epsilon k\frac{r_2}{2}\sin(\Theta)\\
        \\
        \dot{r}_2 = \frac{\epsilon}{8} r_2 \left( 4 - r_2^2 \right) - \epsilon k\frac{r_1}{2}\sin(\Theta) \\
        \\
      \dot{\Theta} = \epsilon \frac{k}{2}\left(  \frac{r_2}{r_1} - \frac{r_1}{r_2} \right)\cos(\Theta)
    \end{array}\right.\,
  \end{equation}
  %
  On s'intéresse aux solutions stationnaires. Pour cela, on s'inspire du cas non couplé. Lorsque $k=0$, la configuration stable correspond à $r_1 = r_2 = 2 $
avec $\Theta$ arbitraire.
%
On cherche s'il y a quelque chose de similaire pour $k \neq 0$. Si on pose :
\[r_1 = r_2 = 2 \quad \text{et} \quad \dot{r}_1 = \dot{r}_2 = \dot{\Theta} = 0\]
On trouve effectivement une solution stationnaire à condition que $\Theta = 0$ ou $\Theta = \pi$. Donc il existe au moins deux solutions stationnaires, une où les deux oscillateurs sont en phase, et une autre en anti-phase.
%
\section{Analyse de stabilité linéaire}
%
Pour déterminer la stabilité de ces solutions stationnaires, on peut faire une analyse de stabilité linéaire dans le voisinage des solutions stationnaires. On notera par la suite les solutions stationnaires par un asterisk :
%
\begin{equation*}
    r_1^* = r_2^* = r^* = 2
    \qquad
    \Theta_1^* = 0
    \qquad
    \Theta_2^* = \pi
\end{equation*}
%
Nous donnant deux points stationnaires dans l'espace de phase $(r_1, r_2, \Theta)$ :
%
\begin{equation}
    \vb{x}_1^* = (r^*, r^*, \Theta_1^*)
    \qquad
    \vb{x}_2^* = (r^*, r^*, \Theta_2^*)
\end{equation}
%
On procède en se plaçant à une petite perturbation près de la solution stationnaire :
%
\begin{equation}
    r_1(t) = r^* + \delta r_1(t)
\end{equation}
%
On prend la derivée temporelle et on prend l'approximation linéaire de $\dot{r}_1$, $\dot{r}_2$ et $\dot{\Theta}$ dans le voisinage de la solution stationnaire $\dot{\vb{x}}^*_i$ :
%
\begin{dmath}
    \delta \dot{r}_1 = \dot{r}_1(r_1, r_2, \Theta)
    = \dot{r}_1(\vb{x}_i^*) 
    + (r_1 - r^*)\pdv{\dot{r}_1}{r_1} + (r_2 - r^*)\pdv{\dot{r}_1}{r_2} + (\Theta - \Theta_i^*)\pdv{\dot{r}_1}{\Theta}
\end{dmath}
%
Sous forme vectorielle on a :
%
\begin{equation}
    \dot{\delta \vb{x}} = M \delta \vb{x}
\end{equation}
%
Avec,
%
\begin{equation}
    \delta \vb{x} = 
    \begin{pmatrix}
        \delta r_1 \\
        \delta r_2 \\
        \delta \Theta \\
    \end{pmatrix}
    \qquad
    \dot{\delta \vb{x}} =
    \begin{pmatrix}
        \dot{\delta r_1} \\
        \dot{\delta r_2} \\
        \dot{\delta \Theta}
    \end{pmatrix}
    \qquad
    M =
    \left. \begin{pmatrix}
        \pdv{\dot{r}_1}{r_1} & \pdv{\dot{r}_1}{r_2} & \pdv{\dot{r}_1}{\Theta} \\
        \\
        \pdv{\dot{r}_2}{r_1} & \pdv{\dot{r}_2}{r_2} & \pdv{\dot{r}_2}{\Theta} \\
        \\
        \pdv{\dot{\Theta}}{r_1} & \pdv{\dot{\Theta}}{r_2} & \pdv{\dot{\Theta}}{\Theta}
    \end{pmatrix}\right|_{\vb{x}_i^*}
    \label{eq:slow_flow}
\end{equation}
%
$M_i$ étant la matrice jacobienne du champ vectorielle $(\dot{r}_1, \dot{r}_1, \dot{\Theta})$ évalué au point stationnaire $\vb{x}^*_i$.
%
Cette approximation linéaire nous permet de déduire l'évolution du système dans le voisinage des points stationnaires de manière qualitatif, notamment si les points sont stables ou instables. 
A moins que l'approximation linéaire soit nulle, la stabilité du système sera inchangé par les termes des ordres plus hauts.
%
%UN SKETCH DE L'ESPACE DE PHASE POURRAIT ÊTRE SYMPA
%On s'intéresse particulièrement à des trajectoires rectilignes 'spéciales' A FINIR
%
%
Dans le cas général, il est possible d'écrire $\delta\vb{x}$ sous forme d'une combinaison linéaire des vecteurs propres de $M$ :
%
\[ \delta\vb{x} = c_1 e^{\lambda_1 t}\vb{v}_1 + c_2 e^{\lambda_2 t}\vb{v}_2 + c_3 e^{\lambda_3 t}\vb{v}_3\]
%
Avec $\vb{v_i}$ et $\lambda_i$ les vecteurs et valeurs propres correspondants, qui ne se ni strictement réel ni strictement distinct l'un des autres.
%
Pour un point stationnaire donné, si $\lambda_i < 0$ pour $j = 1, 2, 3$ alors, on voit que la perturbation $\delta\vb{x}$ décroit avec le temps, donc le point stationnaire est stable. Sinon, le point est instable.
%
On s'intéresse donc surtout aux signes des valeurs propres de $M$.
%
On évalue la matrice jacobienne au deux points stationnaires.
%
\begin{equation}
    M_1 = \begin{pmatrix}
             -\epsilon & 0 & +\epsilon k \\
             0 & -\epsilon & -\epsilon k \\
            -\epsilon k/2 & +\epsilon k/2 & 0
         \end{pmatrix}
    \qquad
    M_2 = \begin{pmatrix}
             -\epsilon & 0 & -\epsilon k \\
             0 & -\epsilon & +\epsilon k \\
             +\epsilon k/2 & -\epsilon k/2 & 0
         \end{pmatrix}
\end{equation}
%
% À $\vb{x}_1^*=(2, 2, 0)$ :
% \begin{equation}
%     M_1 = \begin{pmatrix}
%         -\epsilon & 0 & +\epsilon k \\
%         0 & -\epsilon & -\epsilon k \\
%         -\epsilon k/2 & +\epsilon k/2 & 0
%     \end{pmatrix}
% \end{equation}
% À $\vb{x}_2^*=(2, 2, \pi)$ :
% \begin{equation}
%     M_2 = \begin{pmatrix}
%         -\epsilon & 0 & -\epsilon k \\
%         0 & -\epsilon & +\epsilon k \\
%         +\epsilon k/2 & -\epsilon k/2 & 0
%     \end{pmatrix}
% \end{equation}
%
Les valeurs propres sont déterminées à partir de la condition :
%
\begin{dmath}
    \det(M_i - \lambda I)=0
\end{dmath}
%
\begin{dmath}
    \begin{vmatrix}
        -\epsilon - \lambda & 0 & +\epsilon k \\
        0 & -\epsilon - \lambda & -\epsilon k \\
        -\epsilon k/2 & +\epsilon k/2 & - \lambda
    \end{vmatrix}
    %(-\epsilon - \lambda) \begin{vmatrix}
    %    -\epsilon - \lambda & -k \\
    %    k/2 & - \lambda
    %\end{vmatrix}
    %+ k \begin{vmatrix}
    %    0 & -\epsilon - \lambda \\
    %    -k/2 & k/2
    %\end{vmatrix} \\
    %= (- \epsilon - \lambda)\left( -\lambda(-\epsilon-\lambda) + \frac{k^2}{2} \right) + \frac{k^2}{2}(-\epsilon -\lambda)
    = \epsilon^2 k^2(-\epsilon - \lambda) - \lambda(-\epsilon - \lambda) \\
    = 0
\end{dmath}
%
On trouve la même équation caractéristique pour $M_2$, qui peut être résolu pour obtenir :
%
\begin{equation}
  \left\{\begin{array}{@{}l@{}}
    \lambda_1 = -\epsilon \\
    \\
    \lambda_2 = \frac{1}{2}\left(  -\sqrt{\epsilon^2 - 4\epsilon^2k^2} - \epsilon \right) \\
    \\
    \lambda_3 = \frac{1}{2}\left( \sqrt{\epsilon^2 - 4\epsilon^2k^2} - \epsilon \right)
  \end{array}\right.\,
\end{equation}
%
Il est évident que $\lambda_1$ est strictement négatif. %$\lambda_1 < 0 \ \forall k$
Et pour $0 \leq |k| \leq \frac{1}{2} $ on voit bien que $\sqrt{\epsilon^2 - 4\epsilon^2k^2} \leq \epsilon$, donc :
\[ -\epsilon \leq \lambda_2 \leq -\frac{\epsilon}{2} \qquad -\frac{\epsilon}{2} \leq \lambda_3 \leq 0\]
%
Lorsque $|k| > \frac{1}{2}$, $\lambda_2$ et $\lambda_3$ devienne un couple de complexes conjugués. Géométriquement, cela correspond à des trajectoires spirales oscillant autour du point stationnaire – effectivement, on peut vérifier numériquement que $r_1$, $r_2$ et $\Theta$ oscillent dans ces cas-là. La stabilité est déterminée par la partie réelle des valeurs propres. Or :
\[ \Re(\lambda_2) = \Re(\lambda_3) = -\frac{\epsilon}{2} \]
%
La partie réelle des trois valeurs propres étant toujours négatives, 
on voit donc que dans le cadre de notre approximation, les points stationnaires $(2, 2, 0)$, $(2, 2, \pi)$ sont stables pour tout $k$. 
Ce qui nous montre que les deux oscillateurs peuvent se synchroniser soit en phase ou en anti-phase. 
On n'a pas éliminé la possibilité que l'oscillateur ne se synchronise pas. 
Pour cela, il faudrait étudier de plus près l'étendu des bassins d'attraction - s'il y a des conditions 'neutres' 
qui ne donnent pas lieu à de la synchronisation

Étant donné que les valeurs propres sont identiques pour les deux points stationnaires, 
on s'attend a ce que les deux configurations synchronisées soient aussi stables l'une que l'autre. 
Donc, l'état synchronisé que va choisir le système va dépendre des conditions initiales.

\begin{figure}
    \centering
    % \begin{subcaptionblock}{.95\linewidth}
    %     \includegraphics[width=.35\linewidth]{images/couplee/eigenvalues_ϵ=0.2.png}%
    %     \hfill
    %     \includegraphics[width=.60\linewidth]{images/couplee/smaller_k=1_x0=[0.038,-0.018,0.265,-0.486]_ϵ=0.1.png}%
    % \end{subcaptionblock}
    \subcaptionbox{}[.37\linewidth]{%
        \includegraphics[width=\linewidth]{images/couplee/eigenvalues_ϵ=0.2.png}%
        \label{fig:eigenvals}
    }
    \hfill
    \subcaptionbox{}[.60\linewidth]{%
        \includegraphics[width=\linewidth]{images/couplee/smaller_notitle_k=1_x0=[0.038,-0.018,0.265,-0.486]_ϵ=0.1.png}%
        \label{fig:comparaison}
    }
    \caption{\textbf{(a)} Tracé de la partie réelle des valeurs propres $\lambda_i$ en fonction de $k$, $\epsilon=0.2$. 
        \textbf{(b)} Comparaison graphique du système original \eqref{eq:coupled_vdp} avec le système moyenné \eqref{eq:slow_flow} faite par intégration numérique – $x_1(0)=0.038$, $x_2(0)=0.265$, $\dot{x}_1(0)=-0.018$, $\dot{x}4_2(0)=-0.486$, $\epsilon=0.1$, $k=1$}
    
\end{figure}
%