\chapter{Conclusion}
%
Dans ce rapport, nous avons étudié le comportement de plusieurs systèmes d'oscillateurs faiblement non-linéaires, notamment un système de deux oscillateurs de Van der Pol identiques couplés.
Pour mener cette étude, nous avons utilisé des méthodes perturbatives, telles que la méthode de Linstedt et l'algorithme de moyennement de Krylov-Bogoliubov.

Après avoir dérivé les équations de mouvement lentement variable pour le système d'oscillateurs couplés, nous avons identifié l'existence de deux solutions stationnaires.
La stabilité locale de ces deux points a été vérifiée par une analyse de stabilité linéaire, ce qui nous a permis de 
caractériser les états de synchronisation en phase et en anti-phase.

Cette étude a également permis de mettre en évidence des différences qualitatives entre les oscillateurs non-linéaires et l'oscillateur harmonique, notamment la présence d'états bistable et l'existence de cycles limites.

Finalement, il reste des avenues intéressantes à explorer pour approfondir l'étude finale sur la synchronisation. Parmi celles-ci,
on peut envisager des cas plus généraux, tels que des oscillateurs à fréquence non identiques, des réseaux d'$n$ oscillateurs ou l'introduction de termes supplémentaires pour modéliser des situations réelles.
L'étude des bassins d'attractions dans l'espace des paramètres pour ces cas plus complexes pourrait aussi être une direction de recherche intéressante.