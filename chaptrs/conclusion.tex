\chapter{Conclusion}
%
Dans ce rapport, nous avons étudié le comportement
de plusieurs systèmes d'oscillateurs faiblement non-linéaires, y compris un système de
deux oscillateurs de Van der Pol identique couplés.
Ceci a été fait en employant des méthodes perturbatives, 
tel que la méthode de Linstedt et l'algorithme de moyennement de Krylov-Bogoliubov.

Après avoir dérivé les équations de mouvement lentement variable pour le système 
d'oscillateurs couplés, nous avons déterminé l'existence de deux solutions stationnaires qui,
si stable correspondrait à des états de synchronisation en phase et en anti-phase.
La stabilité locale de ces deux points a ensuite été vérifiée par une analyse de stabilité linéaire.

Cette étude a aussi permis de mettre en évidence les différences qualitatives
que présentent les oscillateurs non-linéaires par rapport à l'oscillateur harmonique.
Tel que la présence d'états bistable et l'existence de cycles limites.

Finalement, il serait envisageable d'approfondir l'étude finale sur la synchronisation en considérant des
cas plus généraux : oscillateurs à fréquence non identiques, réseau de $n$ oscillateurs, termes non-linéaires supplémentaires, etc.
L'étude des bassins d'attractions dans l'espace des paramètres pour ces cas plus complexe pourrait
aussi se révéler intéressante.
%
%Le travail entreprit au cours de ce stage
%
%Ce fut enrichissant pour l'auteur.
%
%L'auteur tire un bilan très positive de ce stage
%
%
%Avec plus de temps
%
%Choses à faire une autre fois :
%- Avec perturbations thermiques
%- Avec detuning
%- A quoi ressemble le bassin d'attraction