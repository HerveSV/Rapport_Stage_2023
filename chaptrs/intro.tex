\chapter{Introduction}
L'oscillateur, système omniprésent dans la nature, 
joue un rôle essentiel dans de nombreuses applications, 
allant des systèmes électriques et mécaniques aux processus biologiques et 
chimiques. Si les oscillateurs linéaires ont fait l'objet de nombreuses 
études approfondies, il existe un autre domaine fascinant et complexe qui 
mérite une attention particulière : celui des oscillateurs non-linéaires. 

Un domaine qui a joué un rôle particulièrement important dans le development 
de nouvelles technologies électroniques au vingtième siècle, tel que la radio, 
le radar et le laser.

Dans ce rapport, nous chercherons a acquérir un compréhension des 
comportements de plusieurs oscillateurs non-linéaires classiques ainsi que 
des méthodes mathématiques fréquemment utilisé pour analyser de tels problèmes. 
Ensuite nous appliquerons ces méthodes dans l'étude d'un système 
de deux oscillateurs non-linéaires couplées.
% En physique, on travaille souvent avec des systèmes linéaires.
% En effet, la linéairité offre de nombreux comfort, tel que le principe de superposition.
% Ce qui rend les problèmes lineaires beacoup plus facile à étudier, pour nombreux on connait les solutions analytiques exactes.
% Pour cette même raison, il à fallut attendre l'invention et la prolifération de l'ordinateur pour que la recherche sur la dynamique non-linéaire s'intensifie \cite{strogatz_nonlinear_2015}.