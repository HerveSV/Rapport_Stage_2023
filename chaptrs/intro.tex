\chapter{Introduction}
L'oscillation, phénomène omniprésent dans la nature, revêt une importance cruciale dans divers domaines de la science et de l'ingénierie. Les oscillateurs, en particulier, jouent un rôle essentiel dans de nombreuses applications, allant des systèmes électriques et mécaniques aux processus biologiques et chimiques. Si les oscillateurs linéaires ont fait l'objet de nombreuses études approfondies, il existe un autre domaine fascinant et complexe qui mérite une attention particulière : celui des oscillateurs non-linéaires.

% \todo{Ajouter des exemples de la micro ou nanomécanique.
% Etoffer un peu plus la biblio théorie et exp sur les oscillateurs NL
% Mieux faire l'état de l'art et l'intro de la problématique de ton stage
% Etoffer l'intro.
% Quel titre à ton stage ?

% Mettre exemple une publi sur le couplage de oscillateurs nanomécaniques couplés ac ex de la synchronisation}
%
Stimulé principalement par les besoins pratiques de l'ingénierie radio,
le domaine des oscillations non-linéaires a connu une phase de croissance importante au début du vingtième siecle \cite{samoilenko_nn_1994}.
%suscité un intérêt particulier  \cite{samoilenko_nn_1994}.
On peut citer le travail novateur de B. van der Pol sur les oscillations de relaxations, 
ainsi que ``la contribution exceptionnelle de l'école d'Andronov et de l'école de Krylov-Bogoliubov'', 
ces derniers étant responsable pour le développement formel de la méthode de moyennement \cite{mira_historical_1997}.
Après 1960, la prolifération de l'ordinateur à grandement contribué à l'explosion du nombres de nouveau résultats \cite{mira_historical_1997}.
Aujourd'hui, l'étude des oscillateurs non-linéaires nous ouvres des possibilités dans le development de nouvelles technologies nano-mécanique, 
tel que des qubits nano-mécaniques \cite{pistolesi_proposal_2021} ou de détecteurs de force ultrasensible \cite{moser_ultrasensitive_2013}.
%ainsi que le development formel de la méthode approximative de moyennement par Krylov et Bogoliubov
%Un domaine qui a jouit d'un intérêt -- pendant la première moitié du 
%vingtième siècle pour leur importance dans le développement de nouvelles 
%technologies électroniques tel que la radio et le laser \cite{strogatz_nonlinear_2015}.
%joué un rôle particulièrement important dans le développement de nouvelles technologies électroniques au vingtième siècle, tel que la radio, le radar et le laser \cite{strogatz_nonlinear_2015}. 
%Et qui continue à jouer un rôle important dans divers domaines scientifiques modernes.
%- développement de la méthode de moyennement par Krylov et Bogoliubov
%- découvertes des cycles de relaxation, et première observations documenté du chaos déterminés par van der Pol
%- nano-nonlinear oscillators for other applications

L'auteur à effectué sont stage au Laboratoire Ondes et Matière d'Aquitaine avec pour
sujet {l'étude théorique de la synchronisation entre oscillateurs non-linéaires}.
Cette synchronisation est particulièrement intéressante dû
à la présence de cycles limites entretenues de manière autonomes et au comportements collectives
qui en résultent. Les applications potentielles de ces phénomènes collectives
incluent la communication radio, la synchronisation d'horloges et le traitement de signal \cite{djorwe_self-organized_2020}.

Dans ce rapport, nous chercherons à acquérir une compréhension des 
comportements de plusieurs oscillateurs non-linéaires classiques ainsi que 
des méthodes mathématiques fréquemment utilisées pour analyser de tels problèmes. 
Finalement, nous appliquerons ces méthodes dans l'étude d'un système 
de deux oscillateurs non-linéaires couplées afin de déterminer si et comment la synchronisation
peut y avoir lieu.


% L'oscillateur, système omniprésent dans la nature, 
% joue un rôle essentiel dans de nombreuses applications, 
% allant des systèmes électriques et mécaniques aux processus biologiques et 
% chimiques. Si les oscillateurs linéaires ont fait l'objet de nombreuses 
% études approfondies, il existe un autre domaine fascinant et complexe qui 
% mérite une attention particulière : celui des oscillateurs non-linéaires. 

% En physique, on travaille souvent avec des systèmes linéaires.
% En effet, la linéairité offre de nombreux comfort, tel que le principe de superposition.
% Ce qui rend les problèmes lineaires beacoup plus facile à étudier, pour nombreux on connait les solutions analytiques exactes.
% Pour cette même raison, il à fallut attendre l'invention et la prolifération de l'ordinateur pour que la recherche sur la dynamique non-linéaire s'intensifie \cite{strogatz_nonlinear_2015}.