\chapter{Introduction}
L'oscillation, phénomène omniprésent dans la nature, joue un rôle crucial dans divers domaines de la science et de l'ingénierie. Les oscillateurs, en particulier, sont essentiels dans de nombreuses applications, des systèmes électroniques et mécaniques aux processus biologiques et chimiques. Si les oscillateurs linéaires ont fait l'objet de nombreuses études approfondies, il existe un autre domaine fascinant et complexe qui mérite une attention particulière : celui des oscillateurs non-linéaires.

Stimulé principalement par les besoins pratiques de l'ingénierie radio,
le domaine de la dynamique non-linéaire a connu une phase de croissance importante au début du vingtième siècle \cite{samoilenko_nn_1994}.
Ce domaine a été marqué par des contributions notables telles que le travail novateur de B. van der Pol sur les oscillations de relaxation, ainsi que la méthode de moyennement développée par Krylov et Bogoliubov \cite{mira_historical_1997}. Depuis les années 1960, l'avènement de l'ordinateur a considérablement accéléré la découverte de nouveaux résultats en dynamique non-linéaire \cite{mira_historical_1997}.

Aujourd'hui, l'étude des oscillateurs non-linéaires ouvre des perspectives très prometteuses dans le développement de nouvelles nanotechnologies, tel que des qubits nanomécaniques \cite{pistolesi_proposal_2021} et des détecteurs de force ultrasensibles \cite{moser_ultrasensitive_2013}. 
Ce stage de recherche, effectué au Laboratoire Ondes et Matière d'Aquitaine, se concentre sur l'analyse de la synchronisation entre oscillateurs non-linéaires, un phénomène particulièrement intéressant en raison de ses applications potentielles en communication radio, en synchronisation d'horloges et en traitement du signal \cite{djorwe_self-organized_2020}.

Dans ce rapport, nous chercherons à acquérir une compréhension des 
comportements de plusieurs oscillateurs non-linéaires classiques ainsi que 
des méthodes mathématiques fréquemment utilisées pour analyser de tels systèmes. 
Enfin, nous appliquerons ces méthodes dans l'étude d'un système 
de deux oscillateurs non-linéaires couplées afin de déterminer les conditions de synchronisation éventuelle.
%ainsi que le development formel de la méthode approximative de moyennement par Krylov et Bogoliubov
%Un domaine qui a jouit d'un intérêt -- pendant la première moitié du 
%vingtième siècle pour leur importance dans le développement de nouvelles 
%technologies électroniques tel que la radio et le laser \cite{strogatz_nonlinear_2015}.
%joué un rôle particulièrement important dans le développement de nouvelles technologies électroniques au vingtième siècle, tel que la radio, le radar et le laser \cite{strogatz_nonlinear_2015}. 
%Et qui continue à jouer un rôle important dans divers domaines scientifiques modernes.
%- développement de la méthode de moyennement par Krylov et Bogoliubov
%- découvertes des cycles de relaxation, et première observations documenté du chaos déterminés par van der Pol
%- nano-nonlinear oscillators for other applications


% L'oscillateur, système omniprésent dans la nature, 
% joue un rôle essentiel dans de nombreuses applications, 
% allant des systèmes électriques et mécaniques aux processus biologiques et 
% chimiques. Si les oscillateurs linéaires ont fait l'objet de nombreuses 
% études approfondies, il existe un autre domaine fascinant et complexe qui 
% mérite une attention particulière : celui des oscillateurs non-linéaires. 

% En physique, on travaille souvent avec des systèmes linéaires.
% En effet, la linéairité offre de nombreux comfort, tel que le principe de superposition.
% Ce qui rend les problèmes lineaires beacoup plus facile à étudier, pour nombreux on connait les solutions analytiques exactes.
% Pour cette même raison, il à fallut attendre l'invention et la prolifération de l'ordinateur pour que la recherche sur la dynamique non-linéaire s'intensifie \cite{strogatz_nonlinear_2015}.